\chapter{Summary}
\section{Summary}
Having finished building the webshop and having had a final look at it, overall I am satisfied with it. As I managed to achieve most of the goals I had set and managed to create a smaller and simpler webshop. Creating this was a challenging element as I have not had to work alone on such a complex project in all my years. However, it was a great help for me to work on a similar project with my friends in EvoCampus. But there I only had to work on a specific part of the project, not on the frontend and backend. It was a completely new experience, because there was a big difference between working with a team of several people where I had to work on a separate part of the backend and working on every part of the program. I felt that I was working on it alone mainly in the design part because now I could only rely on myself and my own ideas. But I'm satisfied with myself because I think I've managed to create a website that meets today's standards and is a site that users want to use. However, I still have a lot to improve to create a really good webshop, but that's not a problem as I'm currently an apprentice in this field and I want to continue to work on it and improve my skills. Writing this thesis helped me to deepen my knowledge in Angular and Node.js. For me, authentication was a completely new part of Angular but I know that I will use it a lot in the future so I have gained useful knowledge. Also, doing this thesis helped me to get a better insight into how exactly a webshop works in the background. Since the website is still in a rudimentary version, it is still working, but I would like to expand it in the future, if my time and knowledge allows me, and create a webshop with all the features.
\section{Future plan}
First of all, I want to expand the registration section, such as the email confirmation message, to ensure that future users will be sure to provide a valid email address. In addition, I would like to strengthen the security with two-step authentication, so that users can enter their phone number at registration and receive a confirmation code by SMS, which they will have to enter during the registration process. Of course, as on all other websites, two-step authentication would be optional. In addition, I would like to add a forgotten password option to the login page so that users could request a new one via email. Since the webshop is dealing with numismatic products, the range of products should be extended to include coin collecting albums, scales, gloves, magnifying glasses or, later, paper coins. For this reason, the current use of simpler labels would no longer be appropriate and a much more serious filter would have to be added to the site. As I mentioned earlier, since the site is not published on the internet, I did not include a payment option, but if I could create a fully featured and fully functional webshop, I would like to include that as well. There are already a number of ways to do this in today's world, such as PayPal, PaySafe, SimplePay or just a standard credit card option. But in addition to these, I also want to offer the possibility of cash on delivery or even delivery to parcel points. Solving online payment methods would not be the easiest task, but it is an essential part of a modern webshop nowadays. However, the solution of delivery to parcel points would not be so difficult, because my website already has a map at the checkout where you have to enter your exact location. If I want to add this functionality, I only need to modify the map component. Since there are people who don't like to register for a site just to make a purchase, I want to make sure that users who are not registered can also make a purchase. In case I manage to create a fully functional webshop that is used by a large user base, I would like to automate the UI tests in the future for easier maintenance of the site.
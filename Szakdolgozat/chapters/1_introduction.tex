\chapter{Bevezetés}

A mai rohanó világunkban az informatika egyre nagyobb teret hódít magának, ez nagyban megkönnyíti az életünket. Az egyik ilyen, gyakran használt informatikai fejlesztés az internetes vásárlás. Az ehhez használt webáruházak hasznosak az üzletek számára, mivel így olyan vásárlókat is elérhetnek, akik különben a nagy távolság miatt nem őket választanák. A webáruházak segítségével a vásárlást kényelmesen, otthonról is meg lehet ejteni, egy-egy termék beszerzéséhez nem szükséges fizikálisan ingázni a boltok között, ezáltal nő a vásárlói élmény. 

A webáruházi vásárlás mára már annyira népszerű lett, hogy egyes üzletláncok applikációin keresztül már a hétvégi bevásárlást is el tudjuk intézni, amivel éves szinten több órát tudunk magunknak megspórolni.

A webáruházat vásárlók széles köre használja, emiatt szükséges a felület könnyű kezelhetősége és átláthatósága. Mivel a webáruházakra nagy kereslet van, az üzletek nem engedhetik meg maguknak, hogy az oldaluk igénytelen legyen. Hiszen hiába használható jól az oldal, ha nem esztétikus, nem fogja a vevők érdeklődését felkelteni. Emellett a fejlesztőknek arra is oda kell figyelniük, hogy a felhasználók többféle webböngészőt is használni fognak - ezek közül a legismertebbek a Google Chrome, Microsoft Edge, Opera, Mozilla Firefox, Safari - tehát gondoskodniuk kell arról, hogy a weboldal minél több böngészőben használható legyen. Egy sikeres webáruház létrehozásához azonban nem elég a széles termékkínálat és az esztétikus megjelenés. Oda kell figyelnünk, hogy a vásárlók képesek legyenek megtalálni a keresett terméket. Ezt kategóriák és szűrők használatával kell segíteni, így a felhasználó képes lesz céltudatosan vásárolni, amivel időt fog megspórolni. Ezáltal, ha az adott oldalt kielégíti a vásárlók igényeit, akkor a későbbiekben is vissza fognak ide térni, vásárolni.

A jövőben remélhetőleg tengernyi felhasználó fogja használni webáruházunk, akik temérdek rendelést fognak leadni, így lesznek olyan termékek, amelyek elfogynak, emellett érkezhetnek új termékek is. Ezek kezelésére szükség lesz adminisztrátorokra. Ezért biztosítani kell plusz funkciókat, amelyeket a normál felhasználók, vásárlók nem használhatnak. Így két fajta felhasználó lesz: a vásárló és az adminisztrátor.

A fentebbiekből is látható, hogy egy modern, megfelelően működő, a vásárlói igényeket kielégítő webáruház létrehozása komoly kihívás elé állítja a fejlesztőket. 

Szakdolgozatom célja egy olyan régiségekkel, azon belül főként numizmatikával foglalkozó webáruház létrehozása, amely a fentebb leírt követelményeknek megfelel. Azért ezt a témát választottam, mert tanulmányaim során a webes alkalmazások készítése keltette fel leginkább az érdeklődésem, emellett magam is gyűjtök régiségeket, így számos alkalommal vásároltam webáruházakból, mégsem találtam olyan oldalt, amely teljes mértékig kielégítette volna felé támasztott szükségleteimet. Mindegyik oldalnak számos hibája volt, gyakran a termékek közötti szűrési lehetőség nem volt elég alapos vagy nem is volt szűrési lehetőségem, így megnehezítették vásárlásomat, kevésbé tudtam céltudatosan választani. 

\chapter{Összefoglalás és jövőbeli tervek}

\section{Összefoglalás}
Miután befejeztem a webáruház készítését és átnéztem még egyszer utoljára összességében meg vagyok vele elégedve. Mivel sikerült a kitűzött célok nagy részét elérni és sikerült egy kisebb és egyszerűbb webáruházat létrehozni. Ennek létrehozása az kihívásokat állított elem mivel eddigi éveim alatt még nem kellett egyedül dolgoznom egy ilyen komplexebb projekten. Viszont nagy segítséget nyújtott számomra hogy barátaimmal EvoCampus projekten belül dolgoztunk hasonló programon. De ott csak egy adott részen kellett dolgoznom nem pedig a frontendel és backendel is. Ez egy teljesen új tapasztalat volt, mert nagy a különbség hogy egy több fős csapattal dolgozni ahol a backendnek is csak egy szeparált részén kellett dolgoznom vagy pedig minden egyes részének a programnak nekem kellett megcsinálni. Azt hogy egyedül dolgozok rajta főként a tervezés részénél éreztem meg, mivel most csak saját magamra és a saját ötleteimre hagyatkozhattam. De meg vagyok magammal elégedve mivel szerintem sikerült olyan weboldalt létrehozni, amely megfelel a mai normáknak és olyan oldal lett, amelyet a felhasználók szívesen használnak. Viszont hogy egy igazán megfelelő webáruházat létre tudjak hozni még sokat, kell fejlődnöm, de ez nem probléma mivel jelenleg is ilyen téren vagyok gyakornok és ezek után is ezzel akarok foglalkozni és gyarapítani tudásomat. a szakdolgozat írása segített abban, hogy elmélyítsem a tudásom az Angular-ban és a Node.js-ben. Számomra egy teljesen új rész volt az Angularban az autentikáció, de tudom, hogy ezt a későbbiekben is sokszor fogom még használni így hasznos tudásra tettem szert. A szakdolgozat készítés továbbá segített abban, hogy jobban bele lássak, hogy pontosan hogy is működik egy webáruház a háttérben. Mivel a weboldal még így is kezdetleges verzióban van hiába, hogy működi azért még szeretném a későbbiekben, ha időm és tudásom engedi kibővíteni és egy ténylegesen minden funkcióval ellátott webáruházat létrehozni belőle.

\section{Jövőbeli tervek}
Először is a regisztrációs rész akarom kibővíteni ilyen például az email visszaigazoló üzenet ezzel garantálva, hogy a jövőbeli felhasználók biztosan érvényes email címet adjanak meg. Ezen kívül még a biztonságot meg szeretném erősíteni két lépcsős autentikációval tehát regisztrációnál lehetőség nyílik a telefonszámát megadnia a felhasználóknak és sms-ben kapnának egy megerősítő kódot, amelyet aztán a regisztráció folyamán meg kell adniuk. Persze mint minden másik weboldalon a két lépcsős autentikáció csak opcionális lehetőség lenne. Emellett a belépési oldalt is ki szeretném egészíteni egy elfelejtett jelszó opcióval ilyenkor emailen keresztül a felhasználók újat kérhetnének. Mivel a webáruház numizmatikai termékekkel foglalkozik így a kínálat is bővítésre szorul ilyen termékek az érme gyűjtő albumok, mérleg, kesztyű, nagyító vagy későbbiekben papírpénzek is. Ez okból a mostani egyszerűbb címkék használata már nem lenne megfelelő így egy jóval komolyabb szűrővel kéne ellátni az oldalt. Mint már korábban említettem mivel az oldal nincs publikálva az internetre ezért fizetési opció nem került bele de amennyiben létretudnék hozni egy teljesmértékben és minden funkcióval ellátott webáruházat ezt is bele szeretném rakni. Ennek megoldására már számtalan módszer létezik a mai világban ilyen például: PayPal, PaySafe, SimplePay vagy csak normáls bankkártyás vásárlási lehetőség. De ezeken kívül lehetőséget akarok kínálni az utánvételes lehetőségre is vagy akár csomagpontokra való szállításra is. Az online fizetési módszerek megoldása nem lenne a legegyszerűbb feladat, de egy mostani modern webáruháznak ez egy elengedhetetlen része. A csomagpontokra való szállítás megoldása viszont már nem lenne, annyira nehéz megoldás ugyanis a weboldalam már rendelkezik a fizetésnél térképpel ahol a pontos helyzetünket kell megadni. Amennyiben ezt a funkciót bele szeretném építeni akkor csak a map komponenst, kell átalakítanom. Mivel vannak olyan emberek, akik nem szeretnek csak egy vásárlás miatt regisztrálni egy oldalra ezért biztosítani akarom azt is, hogy olyan felhasználók is képesek legyenek vásárolni akik, nincsenek regisztrálva. Abban az esetben ha sikerülne egy teljes körűleg működő webáruházat létrehozni amelyet széleskörű felhasználó bázis használ, a UI teszteket szeretném a későbbiekben automatizálni az oldal egyszerűbb karbantartása miatt.
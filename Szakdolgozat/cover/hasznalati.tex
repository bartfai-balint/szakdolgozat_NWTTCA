\pagestyle{empty}

\section*{Adathordozó használati útmutató}

\vskip 1cm

A mellékelt adathordozón a következő jegyzékek és fájlok találhatók.


\begin{itemize}
\item Program mappa, melyen belül található egy Frontend és egy Backend mappa. Ezeken belül találhatóak a forráskódok,
\item Szakdolgozat mappa, ahol a Szakdolgozat LaTeX kódja található,
\item Szakdolgozat.pdf, mely a szakdolgozatot tartalmazza, PDF formátumban,
\end{itemize}

A Frontend futtatásához a \textit{Visual Studio Code} fejlesztőkörnyezetre lesz szükséges. A VSCode tartalmaz egy terminált, ott a \textit{frontend} mappában kell lefuttatni az \textit{npm install} parancsot, ekkor létrejön egy \textit{node\_modules} mappa, amibe letöltődik az összes szükséges fájl. Következő lépésben az \textit{ng serve} paranccsal el is indíthatjuk a programot ami a \textit{localhost:4200}-on fog futni.

A Backend futtatásához úgyszintén a \textit{Visual Studio Code} fejlesztőkörnyezetre lesz szükséges. A VSCode tartalmaz egy terminált, ott a \textit{backend} mappában kell lefuttatni az \textit{npm install} parancsot, ekkor létrejön egy \textit{node\_modules} mappa, amibe letöltődik az összes szükséges fájl. Következő lépésben az \textit{ng serve/npm start} paranccsal el is indíthatjuk a program backend részét amely a \textit{localhost:5000}-en fog futni. A program futtatásához létre kell hozni a MongoDB Atlas-ban egy megfelelő adatbázis szervert.